\section{Motivazioni e obiettivi del progetto}

Il metodo AE--XAD, presentato in \cite{AE_XAD_Arrays}, ha dimostrato di
raggiungere risultati allo stato dell’arte nella rilevazione di anomalie
industriali, sia a livello di classificazione (X-AUC) sia a livello di
localizzazione (IoU, PRO). La combinazione di un encoder ResNet,
un decoder convoluzionale e una procedura di filtraggio adattivo
consente di produrre heatmap coerenti e stabili, rendendo AE--XAD una
soluzione matura e competitiva.

Tuttavia, l’intero framework rimane fortemente dipendente dalla
capacità espressiva dell’encoder, responsabile di estrarre le
rappresentazioni su cui opera il decoder. Nel lavoro originale, tale
encoder è una rete convoluzionale pre-addestrata. Negli ultimi anni,
i Vision Transformer (ViT) hanno mostrato una capacità superiore nel
modellare relazioni non locali e nel catturare strutture globali
dell’immagine, grazie al meccanismo di attenzione multi-testa. 

A partire da queste considerazioni, il nostro progetto non mira a
modificare la logica di AE--XAD, ma a rispondere a una domanda
specifica:

\begin{quote}
\textit{È possibile migliorare la qualità delle mappe di anomalia e la bontà
delle ricostruzioni sostituendo l’encoder CNN di AE--XAD con un
Vision Transformer pre-addestrato, mantenendo invariata l’intera
pipeline?}
\end{quote}

L’obiettivo del progetto è dunque duplice:

\begin{enumerate}
    \item \textbf{Integrare un encoder ViT all’interno di AE--XAD} senza
    alterare il decoder, lo scoring e il processo di generazione delle
    heatmap.
    
    \item \textbf{Valutare sperimentalmente l’impatto della sostituzione
    dell’encoder} su ricostruzione, localizzazione delle anomalie e metriche
    globali, confrontando i risultati con la versione originale basata su CNN.
\end{enumerate}

Questa analisi consente di verificare se l’impiego di un backbone
Transformer, ormai standard in molte applicazioni di visione, possa
portare benefici anche nelle pipeline di anomaly detection basate su
ricostruzione, preservando al tempo stesso la semplicità e
l’interpretabilità del framework AE--XAD.
