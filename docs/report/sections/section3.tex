\section{Il framework AE-XAD}

Il framework AE-XAD (AutoEncoder for eXplainable Anomaly Detection) è un metodo
di anomaly detection basato su ricostruzione, progettato specificamente per
scenari industriali caratterizzati da anomalie rare, eterogenee e difficilmente
annotabili in modo esaustivo \cite{angiulli2025aexad}. A differenza di autoencoder
generici, AE-XAD introduce una pipeline strutturata che integra scelte
architetturali, una funzione di perdita dedicata e un meccanismo di scoring
esplicitamente orientato alla localizzazione delle anomalie.

L’intero framework è concepito per operare in regime few-shot supervisionato,
sfruttando un numero limitato di esempi anomali durante l’addestramento senza
snaturare la natura reconstruction-based del metodo.

\subsection{Architettura del modello}

L’architettura di AE-XAD è composta da tre componenti principali: un encoder
convoluzionale, un decoder asimmetrico e un modulo di decisione basato
sull’analisi statistica dell’errore di ricostruzione.

L’encoder è costituito dai primi blocchi di una Deep CNN
pre-addestrata (ResNet), utilizzata come estrattore di feature. L’encoder
produce una rappresentazione latente sotto forma di feature map con risoluzione
spaziale fissa pari a $28 \times 28$ e 64 canali, scelta per garantire la
compatibilità con il decoder e per preservare una struttura spaziale
sufficientemente dettagliata.

Il decoder AE-XAD presenta una struttura asimmetrica a due rami. Il primo ramo,
non addestrabile, esegue un’operazione di upsampling diretto della feature map
latente e applica una funzione di attivazione \textit{tanh}, producendo una
ricostruzione regolarizzata che cattura la struttura globale dell’immagine.
Il secondo ramo, completamente addestrabile, è composto da una sequenza di
blocchi convoluzionali e di deconvoluzione con attivazione SELU, ed è progettato
per modellare dettagli più fini della ricostruzione.

Le due ricostruzioni vengono fuse mediante una modulazione moltiplicativa,
secondo una formulazione del tipo $b_2 + b_1 \cdot b_2$, dove $b_1$ e $b_2$
rappresentano rispettivamente l’output del ramo non addestrabile e di quello
addestrabile. Questa scelta architetturale consente di enfatizzare le
discrepanze locali rispetto alla normalità, rendendo più evidente l’errore di
ricostruzione in corrispondenza delle regioni anomale.

Lo strato finale del decoder è costituito da una convoluzione seguita da una
funzione di attivazione sigmoide, che produce l’immagine ricostruita nello
spazio RGB.

\subsection{Funzione di perdita AE-XAD}

Un elemento distintivo del framework AE-XAD è la funzione di perdita, progettata
per guidare l’apprendimento in presenza di un numero limitato di esempi
anomali. La loss è definita a livello pixel-wise e combina il contributo dei
pixel normali e dei pixel anomali in modo differenziato.

Indicando con $x$ l’immagine di input, con $\tilde{x}$ la ricostruzione prodotta
dal decoder e con $y_j \in \{0,1\}$ l’etichetta del pixel $j$ (normale o
anomalo), la funzione di perdita è definita come:

\[
\ell(x, y) =
\sum_{j=1}^{D}
\left[
(1 - y_j)
\frac{(x_j - \tilde{x}_j)^2}{(F(x_j) - x_j)^2}
+
\lambda_y y_j
\frac{(F(x_j) - \tilde{x}_j)^2}{(F(x_j) - x_j)^2}
\right],
\]

dove $F(x_j)$ è una funzione di scaling fissata a 2 e $\lambda_y$ è un termine di
normalizzazione proporzionale al numero di pixel anomali presenti nell’immagine.

Questa formulazione consente di mantenere il focus sull’apprendimento della
normalità, limitando al contempo l’influenza dei pochi esempi anomali disponibili
durante l’addestramento. In tal modo, AE-XAD preserva il paradigma
reconstruction-based, evitando che il modello degeneri in un classificatore
supervisionato.

\subsection{Pipeline di scoring e localizzazione}

In fase di inferenza, l’immagine di input viene ricostruita dal decoder e
confrontata con l’originale per ottenere una \textit{reconstruction error map}
$M \in \mathbb{R}^{H \times W}$, definita come la distanza pixel-wise tra input e
ricostruzione.

Per ridurre il rumore ad alta frequenza e migliorare la coerenza spaziale
dell’errore, la mappa $M$ viene sottoposta a un filtraggio gaussiano adattivo.
Successivamente, viene applicata una soglia statistica globale definita come:

\[
T = \mu + \sigma,
\]

dove $\mu$ e $\sigma$ rappresentano rispettivamente la media e la deviazione
standard dei valori della mappa di errore filtrata.

La binarizzazione della mappa consente di ottenere una segmentazione delle
regioni anomale, mentre uno score globale a livello di immagine viene calcolato
aggregando l’errore normalizzato. Questo meccanismo permette di valutare sia la
presenza di anomalie nell’immagine sia la loro localizzazione spaziale.

\subsection{Assunzioni implicite del framework AE-XAD}

Il funzionamento del framework AE-XAD si basa su una serie di assunzioni
implicite, che ne determinano l’efficacia nei contesti industriali considerati.

In primo luogo, si assume che le anomalie producano errori di ricostruzione
spazialmente localizzati e sufficientemente concentrati, in modo da risultare
statisticamente separabili dal rumore di fondo mediante una soglia globale.
Questa assunzione è coerente con la presenza di difetti fisici localizzati sulle
superfici industriali.

In secondo luogo, il decoder e la pipeline di scoring sono progettati per
operare su feature map dotate di una forte struttura spaziale locale, come
quelle prodotte da encoder convoluzionali. L’inductive bias locale delle CNN
favorisce infatti la preservazione di dettagli e discontinuità spaziali,
fondamentali per la localizzazione pixel-wise delle anomalie.

Infine, l’intero framework presuppone una compatibilità strutturale tra encoder,
decoder e meccanismo decisionale. Qualsiasi modifica a uno di questi componenti
può alterare la distribuzione dell’errore di ricostruzione e compromettere
l’efficacia della soglia statistica adottata.
