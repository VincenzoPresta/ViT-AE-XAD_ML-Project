\section{Setup sperimentale}

In questa sezione vengono descritti il dataset utilizzato, l’architettura del modello, la
funzione di loss, le metriche di valutazione e il protocollo sperimentale adottato. Tutti gli
esperimenti sono stati condotti mantenendo invariata la pipeline di AE-XAD, al fine di
analizzare in modo isolato l’impatto della sostituzione dell’encoder convoluzionale con un
Vision Transformer.

\subsection{Dataset e preprocessing}

Gli esperimenti sono stati condotti sul dataset MVTec AD, ampiamente utilizzato per la
valutazione di metodi di anomaly detection in ambito industriale. Il dataset include immagini
di componenti e superfici industriali, accompagnate da maschere pixel-wise che annotano le
regioni anomale nelle immagini di test.

Seguendo il protocollo adottato nel framework AE-XAD, il dataset viene utilizzato in regime
few-shot supervisionato, in cui un numero limitato di campioni anomali è reso disponibile
durante la fase di addestramento. Le immagini sono ridimensionate a una risoluzione uniforme
e sottoposte a un preprocessing coerente con quello previsto dal metodo originale, senza
introdurre trasformazioni aggiuntive che possano alterare la distribuzione spaziale dei
difetti.

\subsection{Architettura del modello e funzione di loss}

Il framework AE-XAD adotta una pipeline di anomaly detection basata su ricostruzione,
composta da un encoder $E(\cdot)$, un decoder $D(\cdot)$ e un modulo di decisione a valle
\cite{angiulli2025aexad}. Dato un input $x \in \mathbb{R}^{H \times W \times C}$, l’encoder
produce una rappresentazione latente $z = E(x)$, che viene utilizzata dal decoder per
ottenere la ricostruzione $\hat{x} = D(z)$.

L’addestramento del modello è guidato da una funzione di loss di ricostruzione definita
a livello pixel-wise, che misura la discrepanza tra l’immagine originale e quella
ricostruita. In forma generale, la loss può essere espressa come:
\[
\mathcal{L}_{rec}(x, \hat{x}) = \frac{1}{HW} \sum_{i,j} \lVert x_{i,j} - \hat{x}_{i,j} \rVert,
\]
dove $(i,j)$ indicano le coordinate spaziali dei pixel.

Nel regime few-shot supervisionato considerato in AE-XAD, un numero limitato di campioni
anomali è reso disponibile durante l’addestramento. Tali campioni contribuiscono alla
funzione obiettivo secondo la formulazione proposta nel framework originale, consentendo
di guidare l’apprendimento senza alterare la natura reconstruction-based del metodo.

In fase di inferenza, la differenza pixel-wise tra input e ricostruzione produce una
\emph{reconstruction error map} $M \in \mathbb{R}^{H \times W}$, definita come:
\[
M_{i,j} = \lVert x_{i,j} - \hat{x}_{i,j} \rVert.
\]
La mappa di errore viene successivamente filtrata per ridurre il rumore ad alta frequenza
e binarizzata tramite una soglia statistica globale definita come:
\[
T = \mu + \sigma,
\]
dove $\mu$ e $\sigma$ rappresentano rispettivamente la media e la deviazione standard dei
valori di $M$.

La localizzazione delle anomalie è infine ottenuta confrontando la mappa binarizzata con
le maschere di ground truth fornite dal dataset. In questo lavoro, l’intera pipeline
descritta viene mantenuta invariata, sostituendo esclusivamente l’encoder convoluzionale
originale con un Vision Transformer, al fine di analizzare l’impatto del diverso inductive
bias sulla distribuzione spaziale dell’errore di ricostruzione.


\subsection{Metriche di valutazione}

Le prestazioni del modello sono valutate utilizzando le metriche previste dal framework
AE-XAD. In particolare, vengono considerate metriche di rilevazione a livello di immagine e
metriche di localizzazione a livello pixel-wise.

La localizzazione delle anomalie è ottenuta applicando una soglia statistica globale basata su
media e deviazione standard ($\mu + \sigma$) alla mappa di errore di ricostruzione filtrata.
Le prestazioni di localizzazione sono quindi misurate confrontando la mappa binarizzata con le
maschere di ground truth fornite dal dataset.

\subsection{Protocollo sperimentale}

Per valutare in modo rigoroso l’impatto della sostituzione dell’encoder, gli esperimenti sono
condotti modificando un solo componente alla volta. In particolare, decoder, funzione di
loss, metriche di valutazione e pipeline di test sono mantenuti invariati rispetto al framework
AE-XAD originale.

Questo protocollo consente di attribuire eventuali variazioni nelle prestazioni osservate
esclusivamente al tipo di encoder utilizzato, permettendo un’analisi controllata e
scientificamente fondata del ruolo dell’inductive bias nelle prestazioni di anomaly detection.
