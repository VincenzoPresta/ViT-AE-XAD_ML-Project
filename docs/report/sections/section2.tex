\section{Fondamenti teorici}

In questa sezione vengono introdotti i concetti teorici necessari a inquadrare il contesto
metodologico di questo lavoro e a chiarire le assunzioni implicite alla base del framework
utilizzato. In particolare, viene discusso il paradigma di anomaly detection basato su
ricostruzione, viene descritto il framework AE-XAD dal punto di vista concettuale e viene
analizzato il ruolo dell’inductive bias nelle architetture di visione, in relazione al
meccanismo di decisione adottato.

\subsection{Anomaly Detection basata su ricostruzione}

Gli approcci di anomaly detection basati su ricostruzione si fondano sull’idea di apprendere
un modello delle sole istanze normali, in modo tale che le anomalie possano essere identificate
come deviazioni rispetto al comportamento appreso. In questo paradigma, un autoencoder viene
addestrato a ricostruire immagini prive di difetti, minimizzando un errore di ricostruzione
calcolato a livello pixel-wise.

In fase di test, la presenza di anomalie si traduce tipicamente in un aumento dell’errore di
ricostruzione nelle regioni difettose, rendendo possibile l’individuazione delle anomalie sia
a livello di immagine sia a livello pixel-wise. L’efficacia di tali approcci dipende pertanto
non solo dalla capacità del modello di rappresentare correttamente la normalità, ma anche
dalla struttura spaziale dell’errore di ricostruzione prodotto.


\subsection{Inductive bias nelle architetture di visione}

Con il termine \emph{inductive bias} si intende l’insieme di assunzioni strutturali che un
modello incorpora a priori, influenzando il modo in cui generalizza a partire da un numero
limitato di esempi. Nel contesto della visione artificiale, l’inductive bias riveste un ruolo
particolarmente rilevante in scenari few-shot, dove la quantità di dati disponibili non è
sufficiente a guidare completamente l’apprendimento.

Le architetture convoluzionali, come quelle impiegate in AE-XAD, incorporano un forte inductive
bias locale, che favorisce la modellazione di pattern spaziali e la produzione di
rappresentazioni gerarchiche sensibili alla localizzazione. Tale caratteristica risulta
naturalmente coerente con un paradigma di anomaly detection basato su errori di ricostruzione
pixel-wise e su una sogliatura statistica globale.

Architetture caratterizzate da un diverso tipo di rappresentazione, orientate alla modellazione
di relazioni globali tra regioni dell’immagine, possono invece produrre distribuzioni
dell’errore di ricostruzione più diffuse e meno concentrate. In un framework come AE-XAD, in
cui la decisione finale dipende dall’applicazione di una soglia globale a una mappa di errore
spaziale, tale differenza rappresentazionale può tradursi in un disallineamento tra le
assunzioni della pipeline di decisione e le proprietà dell’encoder utilizzato.
