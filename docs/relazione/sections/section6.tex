\section{Inference e generazione delle heatmap}

Durante la fase di inference, AE--XAD utilizza l’errore di ricostruzione per
individuare le regioni anomale dell’immagine. Il processo, descritto in
\cite{AE_XAD_Arrays}, si articola in tre passaggi principali:
calcolo dell’errore, normalizzazione e filtraggio tramite una finestra
gaussiana adattiva.

\subsection{Errore di ricostruzione}

Dato un test sample $t$, si calcola la ricostruzione $\tilde{t}$ ottenuta
dall’Autoencoder e si definisce il vettore di errore:

\[
e = (t - \tilde{t})^2.
\]

L’errore contiene valori elevati nelle regioni in cui il modello non ha
tentato di replicare fedelmente l’input, tipicamente corrispondenti alle zone
anomale.

\subsection{Normalizzazione}

Per rendere l'errore confrontabile attraverso pixel e immagini, AE--XAD
applica la normalizzazione seguendo la procedura definita in Eq.~(3):

\[
\tilde{e} = \frac{e - \mu_e}{\sigma_e},
\]

dove $\mu_e$ e $\sigma_e$ sono media e deviazione standard dei valori in $e$.
La mappa normalizzata $\tilde{e}$ funge da \emph{raw heatmap}, evidenziando in
modo preliminare le regioni devianti.

\subsection{Selezione automatica del filtro gaussiano}

Per migliorare la coerenza spaziale della heatmap, viene applicato un filtro
gaussiano $F_{k}$ di dimensione $(2k+1)\times(2k+1)$. La dimensione $k$ non
è fissata a priori, ma viene stimata automaticamente da AE--XAD sulla base
delle proprietà geometriche delle anomalie. In particolare \cite{AE_XAD_Arrays}:

\begin{enumerate}
    \item si considera la mappa binarizzata di $\tilde{e}$ utilizzando la
    soglia $\mu_{\tilde{e}} + \sigma_{\tilde{e}}$;
    \item si analizzano in tale mappa le componenti connesse per stimare
    l’estensione media delle regioni anomale, orizzontalmente e verticalmente;
    \item il valore $k$ viene scelto come metà della maggiore tra tali
    estensioni.
\end{enumerate}

In questo modo, il filtro si adatta alle dimensioni effettive del difetto,
evitando sia oversmoothing su anomalie piccole sia dispersione su anomalie
estese.

\subsection{Heatmap filtrata e binarizzazione}

Applicando il filtro gaussiano, si ottiene la heatmap finale:

\[
h = F_{k}(\tilde{e}).
\]

Se è richiesta una segmentazione binaria delle anomalie,
la heatmap viene ulteriormente sogliata usando:

\[
\text{threshold} = \mu_h + \sigma_h,
\]

dove $\mu_h$ e $\sigma_h$ sono media e deviazione standard dei valori in $h$.

\subsection{Anomaly score}

AE--XAD definisce un anomaly score più robusto rispetto alla norma
dell’errore grezzo, sfruttando la coerenza spaziale enfatizzata dal filtro:

\[
S(t) = \| e \cdot F_k(e) \|.
\]

Questa formulazione privilegia regioni in cui l’errore è consistente e
spazialmente continuo, riducendo l’impatto del rumore isolato e migliorando
la capacità di rilevare difetti piccoli ma strutturati.
