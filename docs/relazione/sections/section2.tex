
\section{Formulazione del problema}

Sia un insieme di training
$X = \{x_1, \dots, x_n\}$, dove ciascuna osservazione
$x_i \in [0,1]^D$ rappresenta un'immagine normalizzata.
Per ogni immagine è inoltre disponibile una heatmap
binaria $y_i \in \{0,1\}^D$, che specifica per ciascun
pixel se esso sia normale ($y_{i,j}=0$) oppure anomalo
($y_{i,j}=1$). Come indicato in \cite{papers/AE_XAD_Arrays},
una heatmap assume valore zero ovunque per gli esempi
\textit{inlier}, mentre presenta almeno un pixel attivo
per gli esempi \textit{outlier}.

La quantità $\|y_i\|_1$ (somma degli elementi della heatmap)
permette di distinguere formalmente:
\begin{itemize}
    \item gli \textbf{inlier}, caratterizzati da $\|y_i\|_1 = 0$;
    \item gli \textbf{outlier}, per cui $\|y_i\|_1 > 0$.
\end{itemize}
Sia inoltre $I$ l'insieme degli indici degli inlier e $O$ quello
degli outlier, come definito nel paper.

Il compito del modello, dato un insieme di immagini di test
$T = \{t_1, \dots, t_m\}$, consiste nel generare per ciascuna
immagine una heatmap $h_i$ che stimi il contributo di ogni
pixel all’anomalia complessiva. La heatmap può essere:
\begin{itemize}
    \item \textbf{binaria}, con $h_{i,j} \in \{0,1\}$;
    \item \textbf{continua}, tipicamente $h_{i,j} \in [0,1]$,
          indicando il grado di outlierness del pixel.
\end{itemize}

Come descritto in \cite{AE_XAD_Arrays}, la dimensione del
dato $D$ coincide con $H \times W \times C$, ovvero le
dimensioni spaziali e il numero di canali dell'immagine.
L'obiettivo finale è definire per ogni immagine un valore
di \emph{anomaly score} $S(t_i)$, idealmente proporzionale
alla presenza e all'estensione delle regioni anomale.

Il problema posto da AE--XAD può quindi essere formalizzato
come segue:
\begin{quote}
    dato un insieme limitato di esempi anomali annotati a livello pixel,
    addestrare un modello capace di ricostruire fedelmente le regioni
    normali e di enfatizzare, attraverso la ricostruzione, le regioni
    anomale, così da ottenere una heatmap interpretabile che evidenzi
    i pixel maggiormente responsabili della devianza.
\end{quote}

Questo scenario rientra nel paradigma della
\textit{semi-supervised anomaly detection}, poiché il modello sfrutta
informazioni supervisionate limitate (le heatmap anomale sparse) e,
al tempo stesso, apprende la distribuzione dei pixel normali da
grandi quantità di dati privi di difetti.