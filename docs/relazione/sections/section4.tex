
\section{Loss Function}

La componente centrale di AE--XAD è una loss semi-supervisionata progettata
per ottenere una ricostruzione accurata delle regioni normali e, al tempo
stesso, una ricostruzione deliberatamente distante nelle regioni anomale.
Come mostrato in \cite{AE_XAD_Arrays}, dato un campione $x \in [0,1]^D$
e la relativa heatmap binaria $y \in \{0,1\}^D$, la loss per-pixel è definita come:

\begin{equation}
\ell(x,y) =
\sum_{j=1}^{D}
\left[
(1-y_j)\,
\frac{(x_j - \tilde{x}_j)^2}{(F(x_j)-x_j)^2}
\;+\;
\lambda_y\, y_j\,
\frac{(F(x_j)-\tilde{x}_j)^2}{(F(x_j)-x_j)^2}
\right].
\label{eq:pixel_loss}
\end{equation}

La loss totale sull'intero training set è data da:

\begin{equation}
L(X,Y) = \frac{1}{|X|}
\sum_{(x,y) \in X \times Y}
\ell(x,y).
\label{eq:global_loss}
\end{equation}

\subsection{Interpretazione dei termini}

La formulazione della loss riflette due comportamenti distinti:

\begin{itemize}
    \item \textbf{Pixel normali ($y_j = 0$):}
    \[
    \frac{(x_j - \tilde{x}_j)^2}{(F(x_j)-x_j)^2}.
    \]
    In questo caso, il modello è incentivato a ricostruire fedelmente il pixel
    originale $x_j$. Il denominatore normalizza l’errore affinché i contributi
    siano comparabili tra diversi valori di intensità.

    \item \textbf{Pixel anomali ($y_j = 1$):}
    \[
    \lambda_y \frac{(F(x_j)-\tilde{x}_j)^2}{(F(x_j)-x_j)^2}.
    \]
    Qui il modello non deve ricostruire $x_j$, ma deve avvicinare $\tilde{x}_j$
    al valore trasformato $F(x_j)$, forzando un errore di ricostruzione elevato
    nelle regioni anomale. Questo comportamento è essenziale affinché la
    differenza $|x_j - \tilde{x}_j|$ diventi un indicatore affidabile di anomalia.
\end{itemize}

\subsection{Ruolo della trasformazione $F$}
La funzione $F : [0,1] \to \mathbb{R}$ è un iperparametro introdotto per
massimizzare la distanza di ricostruzione nelle regioni anomale.  
Due scelte discusse in \cite{AE_XAD_Arrays} sono:

\begin{itemize}
    \item $F_{-}(x) = 1 - x$, che rappresenta il negativo del pixel;
    \item $F_{v}(x) = v$, con $v \notin [0,1]$ (nel paper viene utilizzato $v = 2$).
\end{itemize}

In entrambi i casi, $F(x_j)$ è costruita per essere distante dall'originale
$x_j$, aumentando così l’ampiezza dell’errore ricostruttivo nelle zone anomale.

\subsection{Peso di bilanciamento $\lambda_y$}
Il coefficiente $\lambda_y$ compensa lo sbilanciamento tra pixel normali e
anomali:

\[
\lambda_y =
\begin{cases}
D / \|y\|_1 & \text{se } \|y\|_1 > 0,\\
1 & \text{altrimenti}.
\end{cases}
\]

Poiché le anomalie sono generalmente molto rare nello spazio dei pixel,
$\lambda_y$ evita che la loss sia dominata dai termini relativi ai pixel normali.

\subsection{Effetto complessivo della loss}

Complessivamente, la loss in Eq.~\eqref{eq:pixel_loss} realizza un comportamento
\emph{explainability-by-design}: le regioni normali vengono ricostruite con
accuratezza, mentre quelle anomale vengono ricostruite in modo intenzionalmente
errato. La differenza tra immagine originale e ricostruita diventa così una
stima affidabile dell’outlierness locale, utilizzabile per generare heatmap
interpretabili.
